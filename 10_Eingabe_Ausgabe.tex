\documentclass[xelatex,aspectratio=169]{beamer}

\hfuzz=10pt
\vfuzz=10pt

% Theme
\usetheme{htw}
\setbeamertemplate{navigation symbols}{}
\setbeamertemplate{theorems}[numbered]
\setbeamercovered{transparent}

%\logo{\includegraphics[height=0.5cm]{HTWD_color.png}}

% Packages
\usepackage{polyglossia}
\setmainlanguage{german}
\setotherlanguage{english}
\usepackage{newunicodechar}
\newfontfamily\symbolfont{Noto Color Emoji}
\usepackage{diagbox}

\usepackage[bigfiles]{pdfbase}
\ExplSyntaxOn
\NewDocumentCommand\embedvideo{smm}{
\group_begin:
\leavevmode
\tl_if_exist:cTF{file_\file_mdfive_hash:n{#3}}{
    \tl_set_eq:Nc\video{file_\file_mdfive_hash:n{#3}}
}{
    \IfFileExists{#3}{}{\GenericError{}{File~`#3'~not~found}{}{}}
    \pbs_pdfobj:nnn{}{fstream}{{}{#3}}
    \pbs_pdfobj:nnn{}{dict}{
        /Type/Filespec/F~(#3)/UF~(#3)
        /EF~<</F~\pbs_pdflastobj:>>
    }
    \tl_set:Nx\video{\pbs_pdflastobj:}
    \tl_gset_eq:cN{file_\file_mdfive_hash:n{#3}}\video
}
%
\pbs_pdfobj:nnn{}{dict}{
    /Type/RichMediaInstance/Subtype/Video
    /Asset~\video
    /Params~<</FlashVars (
    source=#3&
    skin=SkinOverAllNoFullNoCaption.swf&
    skinAutoHide=true&
    skinBackgroundColor=0x5F5F5F&
    skinBackgroundAlpha=0.75
    )>>
}
%
\pbs_pdfobj:nnn{}{dict}{
/Type/RichMediaConfiguration/Subtype/Video
/Instances~[\pbs_pdflastobj:]
}
%
\pbs_pdfobj:nnn{}{dict}{
/Type/RichMediaContent
/Assets~<<
/Names~[(#3)~\video]
>>
/Configurations~[\pbs_pdflastobj:]
}
\tl_set:Nx\rmcontent{\pbs_pdflastobj:}
%
\pbs_pdfobj:nnn{}{dict}{
    /Activation~<<
    /Condition/\IfBooleanTF{#1}{PV}{XA}
    /Presentation~<</Style/Embedded>>
    >>
    /Deactivation~<</Condition/PI>>
}
%
\hbox_set:Nn\l_tmpa_box{#2}
\tl_set:Nx\l_box_wd_tl{\dim_use:N\box_wd:N\l_tmpa_box}
\tl_set:Nx\l_box_ht_tl{\dim_use:N\box_ht:N\l_tmpa_box}
\tl_set:Nx\l_box_dp_tl{\dim_use:N\box_dp:N\l_tmpa_box}
\pbs_pdfxform:nnnnn{1}{1}{}{}{\l_tmpa_box}
%
\pbs_pdfannot:nnnn{\l_box_wd_tl}{\l_box_ht_tl}{\l_box_dp_tl}{
    /Subtype/RichMedia
    /BS~<</W~0/S/S>>
    /Contents~(embedded~video~file:#3)
    /NM~(rma:#3)
    /AP~<</N~\pbs_pdflastxform:>>
    /RichMediaSettings~\pbs_pdflastobj:
    /RichMediaContent~\rmcontent
}
\phantom{#2}
\group_end:
}
\ExplSyntaxOff


\usepackage{graphicx}
\usepackage[export]{adjustbox}
\usepackage{animate}
%\usepackage[dvipdfmx]{movie15_dvipdfmx}
\usepackage{media9}
\usepackage{tabularx}
\usepackage{colortbl}
\usepackage{booktabs}
\usepackage{makecell}
\usepackage{ltablex}
\usepackage{array}
\usepackage{multirow}
\usepackage{amsmath}
\usepackage{amsthm}
\usepackage{mathtools}

\DeclarePairedDelimiter\ceil{\lceil}{\rceil}
\DeclarePairedDelimiter\floor{\lfloor}{\rfloor}
%\renewcommand{\arraystretch}{1.5}
\newcolumntype{L}[1]{>{\raggedright\let\newline\\\arraybackslash\hspace{0pt}}p{#1}}
\newcolumntype{C}[1]{>{\centering\let\newline\\\arraybackslash\hspace{0pt}}p{#1}}
\newcolumntype{R}[1]{>{\raggedleft\let\newline\\\arraybackslash\hspace{0pt}}p{#1}}
%\renewcommand\thesatz{\arabic{section}.\arabic{theorem}}
\makeatletter
\@addtoreset{theorem}{lecture}
\makeatother

\newtheorem{satz}{Satz}[section]
\newtheorem{lem}{Lemma}[section]
\newtheorem{beh}{Behauptung}[section]
\newtheorem{define}{Definition}[section]
\numberwithin{equation}{section}
\usepackage{ragged2e}
\usepackage{etoolbox}

\usepackage{color}
\usepackage{colortbl}
\definecolor{hellgrau}{rgb}{0.85,0.85,0.85}
\definecolor{hellrot}{rgb}{1,0.7,0.7}

\usepackage{tikz}
\usetikzlibrary{shapes,arrows.meta,calc,arrows,positioning,patterns,tikzmark,overlay-beamer-styles}
%\usepackage{tikz-uml}
\usepackage{pgfplots}  % for elliptic curves (part 8)
\pgfplotsset{compat=1.18}
\usepackage{pgffor}
\usepackage{pgfmath-xfp}
\tikzset{>=latex}
\tikzset{
    invisible/.style={opacity=0},
    visible on/.style={alt={#1{}{invisible}}},
    alt/.code args={<#1>#2#3}{%
            \alt<#1>{\pgfkeysalso{#2}}{\pgfkeysalso{#3}} % \pgfkeysalso doesn't change the path
        },
}

\usepackage{paralist}

\usepackage{url}
\def\UrlBreaks{\do\/\do-}
\PassOptionsToPackage{hyphens}{url}\usepackage{hyperref}

\usepackage[normalem]{ulem} % gestrichelte Unterstreichung (\dashuline{})
\usepackage{cancel}

\makeatletter
\renewcommand{\itemize}[1][]{%
    \beamer@ifempty{#1}{}{\def\beamer@defaultospec{#1}}%
    \ifnum \@itemdepth >2\relax\@toodeep\else
        \advance\@itemdepth\@ne
        \beamer@computepref\@itemdepth% sets \beameritemnestingprefix
        \usebeamerfont{itemize/enumerate \beameritemnestingprefix body}%
        \usebeamercolor[fg]{itemize/enumerate \beameritemnestingprefix body}%
        \usebeamertemplate{itemize/enumerate \beameritemnestingprefix body begin}%
        \list
        {\usebeamertemplate{itemize \beameritemnestingprefix item}}
        {\def\makelabel##1{%
                {%
                        \hss\llap{{%
                                    \usebeamerfont*{itemize \beameritemnestingprefix item}%
                                    \usebeamercolor[fg]{itemize \beameritemnestingprefix item}##1}}%
                    }%
            }%
        }
    \fi%
    \beamer@cramped%
    \justifying% NEW
    %\raggedright% ORIGINAL
    \beamer@firstlineitemizeunskip%
}
\makeatother

\apptocmd{\frame}{}{\justifying}{}

\renewcommand\theadfont{\bfseries\sffamily}
\usepackage{ragged2e}
\usepackage{newpxtext}

\setsansfont{opensans}[
    Scale=MatchLowercase,
    UprightFont=*-regular,
    BoldFont=*-bold,
    ItalicFont=*-italic,
    BoldItalicFont=*-bolditalic,
]

% Title
\usepackage[usetransparent=false]{svg}
% Import references
\usepackage[backend=biber,style=numeric,sorting=none]{biblatex}
\addbibresource{references.bib}

%\AtBeginSection[]{
%  \begin{frame}
%    \vfill
%    \centering
%    \begin{beamercolorbox}[sep=8pt,center,shadow=true,rounded=true]{title}
%      \usebeamerfont{title}\thesection.~\secname\par%
%    \end{beamercolorbox}
%    \vfill
%  \end{frame}
%}

\makeatletter
\newenvironment{noheadline}{
    \setbeamertemplate{headline}{}
    \addtobeamertemplate{frametitle}{\vspace*{-0.9\baselineskip}}{}
}{}
\makeatother


\usepackage{xcolor}
\usepackage{algorithm}
\usepackage[linesnumbered,ruled,lined,commentsnumbered,algo2e,ngerman,ngermankw]{algorithm2e}
\usepackage{algorithmic}
\usepackage{caption}
\usepackage[newfloat]{minted}
\captionsetup[listing]{position=top}
\definecolor{mintedbg}{HTML}{282828}
\setminted{
    breaklines=true,
    bgcolor=mintedbg,
    style=monokai,
    formatcom=\color{white}
}
\usepackage{etoolbox}
\makeatletter
% replace \medskip before and after the box with nothing, i.e., remove it
\patchcmd{\minted@colorbg}{\medskip}{}{}{}
\patchcmd{\endminted@colorbg}{\medskip}{}{}{}
\makeatother

\renewcommand{\theFancyVerbLine}{\textcolor{black}{\arabic{FancyVerbLine}}}

\usepackage{pifont}
\newcommand{\cmark}{\ding{51}}%
\newcommand{\xmark}{\ding{55}}%

\newenvironment{changemargin}[2]{%
    \begin{list}{}{%
            \setlength{\topsep}{0pt}%
            \setlength{\leftmargin}{#1}%
            \setlength{\rightmargin}{#2}%
            \setlength{\listparindent}{\parindent}%
            \setlength{\itemindent}{\parindent}%
            \setlength{\parsep}{\parskip}%
        }%
        \item[]}{\end{list}}


\usepackage{csquotes}

% Title
\title{Ein- und Ausgabe von Dateien}
\author{Prof. Dr. Lukas Iffländer}
\institute{HTW Dresden}
\date{}
\usepackage{svg}

% Begin document
\begin{document}

% Title slide
\begin{frame}
    \titlepage
\end{frame}

\section{Ein- und Ausgabe von Dateien}

\begin{frame}{Ein- und Ausgabe}
    \begin{columns}
        \begin{column}{0.5\textwidth}
            \begin{itemize}
                \item nichtflüchtig gespeichertes Datenobjekt das von Programmen gelesen bzw. geschrieben werden kann
                \item wird gespeichert im Dateisystem, das oft Teil des Betriebssystems ist (Windows, Linux, IOS, Android),
                \item Vom Dateisystem wird die Zuordnung von Namen, die Platzierung in Verzeichnissen, die Zuordnung von Eigenschaften (Zugriffsdatum und -Zeit, Eigentümer, Berechtigungen u.a.) vorgenommen
            \end{itemize}
        \end{column}
        \begin{column}{0.5\textwidth}
        \end{column}
    \end{columns}
\end{frame}

\begin{frame}{Ein- und Ausgabe}
    \begin{columns}[T]
        \begin{column}{0.5\textwidth}
            \begin{itemize}
                \item nichtflüchtig gespeichertes Datenobjekt das von Programmen gelesen bzw. geschrieben werden kann
                \item wird gespeichert im Dateisystem, das oft Teil des Betriebssystems ist (Windows, Linux, IOS, Android),
                \item Vom Dateisystem wird die Zuordnung von Namen, die Platzierung in Verzeichnissen, die Zuordnung von Eigenschaften (Zugriffsdatum und -Zeit, Eigentümer, Berechtigungen u.a.) vorgenommen
            \end{itemize}
        \end{column}
        \begin{column}{0.5\textwidth}
            \centering
            \includegraphics[height=.7\textheight]{fig/io_basic.pdf}
        \end{column}
    \end{columns}
\end{frame}

\begin{frame}{Dateiinhalt und Strukturierung}
    \begin{itemize}
        \item Datei kann Textdaten (utf-8, ASCII) enthalten und zeilenweise gelesen und geschrieben werden.
        \item Inhalt und Struktur sind Programmierern oder Erstellern der Datei überlassen. Strukturen sind oft vorgegeben, um den Austausch von Daten zu erleichtern.
        \item Daten können metastrukturiert sein, z.B. in Form von Tabellen, JSON, XML, YAML.
        \item Datei kann auch Daten im Binärformat enthalten, z.B. Bilder, Audiodateien, Videos, ausführbare Dateien.
    \end{itemize}

\end{frame}

\begin{frame}{Zeilenweises Einlesen und Schreiben}{Beispiel I}
    \inputminted[firstline=6,lastline=19]{python}{src/io_example.py}
\end{frame}

\begin{frame}{Zeilenweises Einlesen und Schreiben}{Beispiel II}
    \inputminted[firstline=26,lastline=37]{python}{src/io_example.py}
\end{frame}

\begin{frame}{Zeilenweises Einlesen und Schreiben}{Beispiel III}
    \inputminted[firstline=43,lastline=52]{python}{src/io_example.py}
\end{frame}

\begin{frame}{Schreiben von Textdateien}
    \begin{columns}
        \begin{column}{0.5\textwidth}
            \inputminted[firstline=1,lastline=9]{python}{src/io_write_example.py}
        \end{column}
        \begin{column}{0.5\textwidth}
            Daten können in Textdateien geschrieben werden, wenn Elemente in String-Qbjekte umgewandelt werden können

            \begin{block}{Access-Modi}
                \begin{itemize}
                    \item \texttt{w} - Schreiben, Datei wird überschrieben
                    \item \texttt{a} - Anhängen, Datei wird erweitert
                    \item \texttt{r} - Lesen, Datei muss existieren
                    \item \texttt{t} - Textmodus, Standardmodus
                \end{itemize}

            \end{block}
        \end{column}
    \end{columns}

\end{frame}

\begin{frame}{Schreiben von Textdatein}{Aus Liste}
    \inputminted[firstline=15,lastline=26]{python}{src/io_write_example.py}
\end{frame}

\begin{frame}{Lesen und Schreiben von JSON und XML}
    JSON (JavaScript Object Notation) und XML (eXtensible Markup Language) sind Formate zur strukturierten Speicherung von Daten.

    \begin{itemize}
        \item Sie werden als Textdateien gespeichert und übertragen, Kodierung ist meist UTF-8.
        \item Daten sind durch Klammerung (JSON) oder durch Tags (XML) strukturiert.
        \item Elemente können hierarchisch angeordnet sein. Listen und Dictionaries (JSON) oder Elemente mit Attributen (XML) sind möglich.
    \end{itemize}
\end{frame}

\begin{frame}{JSON-Format}
    Beispiel für in JSON kodierte Strings:
    Array in JSON:
    \inputminted{JSON}{src/io_json_example1.json}

    Key-Value in JSON:
    \inputminted{JSON}{src/io_json_example2.json}

    Key-Value in JSON mit Liste:
    \inputminted{JSON}{src/io_json_example3.json}

\end{frame}

\begin{frame}{JSON IO}{Beispiel -- Datenstruktur}
    \inputminted[firstline=1,lastline=14]{python}{src/io_json_io.py}
\end{frame}
\begin{frame}{JSON IO}{Beispiel -- Schreiben}
    \small
    \inputminted[firstline=19,lastline=34]{python}{src/io_json_io.py}
\end{frame}
\begin{frame}{JSON IO}{Beispiel -- Lesen I}
    \small
    \inputminted[firstline=35,lastline=49]{python}{src/io_json_io.py}
\end{frame}
\begin{frame}{JSON IO}{Beispiel -- Lesen II}
    \small
    \inputminted[firstline=51,lastline=58]{python}{src/io_json_io.py}
\end{frame}

\begin{frame}{XML-Format}
    \begin{columns}
        \begin{column}{0.5\textwidth}
            \inputminted{xml}{src/waren_short.xml}
        \end{column}
        \begin{column}{0.5\textwidth}
            Speicherung von Daten in XML:
            \begin{itemize}
                \item Als Inhalt von Elementen: \mintinline{xml}|<element>inhalt</element>|
                \item Als Attribute von Elementen: \mintinline{xml}|<element attribut="wert"/>|
            \end{itemize}
        \end{column}
    \end{columns}
\end{frame}

\begin{frame}{XML IO}{Beispiel -- Lesen}
    \small
    \inputminted[firstline=1,lastline=17]{python}{src/io_xml.py}
\end{frame}

\begin{frame}{XML IO}{Beispiel -- Schreiben (Root-Element)}
    \small
    \inputminted[firstline=19,lastline=21]{python}{src/io_xml.py}
\end{frame}

\begin{frame}{XML IO}{Beispiel -- Schreiben (Daten)}
    \smaller
    \inputminted[firstline=23,lastline=40]{python}{src/io_xml.py}
\end{frame}

\begin{frame}{XML IO}{Beispiel -- Schreiben (File)}
    \small
    \inputminted[firstline=42,lastline=45]{python}{src/io_xml.py}
\end{frame}

\begin{frame}{NumPy IO}
    \small
    \inputminted[firstline=1,lastline=14]{python}{src/io_numpy.py}
\end{frame}

\end{document}