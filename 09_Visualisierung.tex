\documentclass[xelatex,aspectratio=169]{beamer}

\hfuzz=10pt
\vfuzz=10pt

% Theme
\usetheme{htw}
\setbeamertemplate{navigation symbols}{}
\setbeamertemplate{theorems}[numbered]
\setbeamercovered{transparent}

%\logo{\includegraphics[height=0.5cm]{HTWD_color.png}}

% Packages
\usepackage{polyglossia}
\setmainlanguage{german}
\setotherlanguage{english}
\usepackage{newunicodechar}
\newfontfamily\symbolfont{Noto Color Emoji}
\usepackage{diagbox}

\usepackage[bigfiles]{pdfbase}
\ExplSyntaxOn
\NewDocumentCommand\embedvideo{smm}{
\group_begin:
\leavevmode
\tl_if_exist:cTF{file_\file_mdfive_hash:n{#3}}{
    \tl_set_eq:Nc\video{file_\file_mdfive_hash:n{#3}}
}{
    \IfFileExists{#3}{}{\GenericError{}{File~`#3'~not~found}{}{}}
    \pbs_pdfobj:nnn{}{fstream}{{}{#3}}
    \pbs_pdfobj:nnn{}{dict}{
        /Type/Filespec/F~(#3)/UF~(#3)
        /EF~<</F~\pbs_pdflastobj:>>
    }
    \tl_set:Nx\video{\pbs_pdflastobj:}
    \tl_gset_eq:cN{file_\file_mdfive_hash:n{#3}}\video
}
%
\pbs_pdfobj:nnn{}{dict}{
    /Type/RichMediaInstance/Subtype/Video
    /Asset~\video
    /Params~<</FlashVars (
    source=#3&
    skin=SkinOverAllNoFullNoCaption.swf&
    skinAutoHide=true&
    skinBackgroundColor=0x5F5F5F&
    skinBackgroundAlpha=0.75
    )>>
}
%
\pbs_pdfobj:nnn{}{dict}{
/Type/RichMediaConfiguration/Subtype/Video
/Instances~[\pbs_pdflastobj:]
}
%
\pbs_pdfobj:nnn{}{dict}{
/Type/RichMediaContent
/Assets~<<
/Names~[(#3)~\video]
>>
/Configurations~[\pbs_pdflastobj:]
}
\tl_set:Nx\rmcontent{\pbs_pdflastobj:}
%
\pbs_pdfobj:nnn{}{dict}{
    /Activation~<<
    /Condition/\IfBooleanTF{#1}{PV}{XA}
    /Presentation~<</Style/Embedded>>
    >>
    /Deactivation~<</Condition/PI>>
}
%
\hbox_set:Nn\l_tmpa_box{#2}
\tl_set:Nx\l_box_wd_tl{\dim_use:N\box_wd:N\l_tmpa_box}
\tl_set:Nx\l_box_ht_tl{\dim_use:N\box_ht:N\l_tmpa_box}
\tl_set:Nx\l_box_dp_tl{\dim_use:N\box_dp:N\l_tmpa_box}
\pbs_pdfxform:nnnnn{1}{1}{}{}{\l_tmpa_box}
%
\pbs_pdfannot:nnnn{\l_box_wd_tl}{\l_box_ht_tl}{\l_box_dp_tl}{
    /Subtype/RichMedia
    /BS~<</W~0/S/S>>
    /Contents~(embedded~video~file:#3)
    /NM~(rma:#3)
    /AP~<</N~\pbs_pdflastxform:>>
    /RichMediaSettings~\pbs_pdflastobj:
    /RichMediaContent~\rmcontent
}
\phantom{#2}
\group_end:
}
\ExplSyntaxOff


\usepackage{graphicx}
\usepackage[export]{adjustbox}
\usepackage{animate}
%\usepackage[dvipdfmx]{movie15_dvipdfmx}
\usepackage{media9}
\usepackage{tabularx}
\usepackage{colortbl}
\usepackage{booktabs}
\usepackage{makecell}
\usepackage{ltablex}
\usepackage{array}
\usepackage{multirow}
\usepackage{amsmath}
\usepackage{amsthm}
\usepackage{mathtools}

\DeclarePairedDelimiter\ceil{\lceil}{\rceil}
\DeclarePairedDelimiter\floor{\lfloor}{\rfloor}
%\renewcommand{\arraystretch}{1.5}
\newcolumntype{L}[1]{>{\raggedright\let\newline\\\arraybackslash\hspace{0pt}}p{#1}}
\newcolumntype{C}[1]{>{\centering\let\newline\\\arraybackslash\hspace{0pt}}p{#1}}
\newcolumntype{R}[1]{>{\raggedleft\let\newline\\\arraybackslash\hspace{0pt}}p{#1}}
%\renewcommand\thesatz{\arabic{section}.\arabic{theorem}}
\makeatletter
\@addtoreset{theorem}{lecture}
\makeatother

\newtheorem{satz}{Satz}[section]
\newtheorem{lem}{Lemma}[section]
\newtheorem{beh}{Behauptung}[section]
\newtheorem{define}{Definition}[section]
\numberwithin{equation}{section}
\usepackage{ragged2e}
\usepackage{etoolbox}

\usepackage{color}
\usepackage{colortbl}
\definecolor{hellgrau}{rgb}{0.85,0.85,0.85}
\definecolor{hellrot}{rgb}{1,0.7,0.7}

\usepackage{tikz}
\usetikzlibrary{shapes,arrows.meta,calc,arrows,positioning,patterns,tikzmark,overlay-beamer-styles}
%\usepackage{tikz-uml}
\usepackage{pgfplots}  % for elliptic curves (part 8)
\pgfplotsset{compat=1.18}
\usepackage{pgffor}
\usepackage{pgfmath-xfp}
\tikzset{>=latex}
\tikzset{
    invisible/.style={opacity=0},
    visible on/.style={alt={#1{}{invisible}}},
    alt/.code args={<#1>#2#3}{%
            \alt<#1>{\pgfkeysalso{#2}}{\pgfkeysalso{#3}} % \pgfkeysalso doesn't change the path
        },
}

\usepackage{paralist}

\usepackage{url}
\def\UrlBreaks{\do\/\do-}
\PassOptionsToPackage{hyphens}{url}\usepackage{hyperref}

\usepackage[normalem]{ulem} % gestrichelte Unterstreichung (\dashuline{})
\usepackage{cancel}

\makeatletter
\renewcommand{\itemize}[1][]{%
    \beamer@ifempty{#1}{}{\def\beamer@defaultospec{#1}}%
    \ifnum \@itemdepth >2\relax\@toodeep\else
        \advance\@itemdepth\@ne
        \beamer@computepref\@itemdepth% sets \beameritemnestingprefix
        \usebeamerfont{itemize/enumerate \beameritemnestingprefix body}%
        \usebeamercolor[fg]{itemize/enumerate \beameritemnestingprefix body}%
        \usebeamertemplate{itemize/enumerate \beameritemnestingprefix body begin}%
        \list
        {\usebeamertemplate{itemize \beameritemnestingprefix item}}
        {\def\makelabel##1{%
                {%
                        \hss\llap{{%
                                    \usebeamerfont*{itemize \beameritemnestingprefix item}%
                                    \usebeamercolor[fg]{itemize \beameritemnestingprefix item}##1}}%
                    }%
            }%
        }
    \fi%
    \beamer@cramped%
    \justifying% NEW
    %\raggedright% ORIGINAL
    \beamer@firstlineitemizeunskip%
}
\makeatother

\apptocmd{\frame}{}{\justifying}{}

\renewcommand\theadfont{\bfseries\sffamily}
\usepackage{ragged2e}
\usepackage{newpxtext}

\setsansfont{opensans}[
    Scale=MatchLowercase,
    UprightFont=*-regular,
    BoldFont=*-bold,
    ItalicFont=*-italic,
    BoldItalicFont=*-bolditalic,
]

% Title
\usepackage[usetransparent=false]{svg}
% Import references
\usepackage[backend=biber,style=numeric,sorting=none]{biblatex}
\addbibresource{references.bib}

%\AtBeginSection[]{
%  \begin{frame}
%    \vfill
%    \centering
%    \begin{beamercolorbox}[sep=8pt,center,shadow=true,rounded=true]{title}
%      \usebeamerfont{title}\thesection.~\secname\par%
%    \end{beamercolorbox}
%    \vfill
%  \end{frame}
%}

\makeatletter
\newenvironment{noheadline}{
    \setbeamertemplate{headline}{}
    \addtobeamertemplate{frametitle}{\vspace*{-0.9\baselineskip}}{}
}{}
\makeatother


\usepackage{xcolor}
\usepackage{algorithm}
\usepackage[linesnumbered,ruled,lined,commentsnumbered,algo2e,ngerman,ngermankw]{algorithm2e}
\usepackage{algorithmic}
\usepackage{caption}
\usepackage[newfloat]{minted}
\captionsetup[listing]{position=top}
\definecolor{mintedbg}{HTML}{282828}
\setminted{
    breaklines=true,
    bgcolor=mintedbg,
    style=monokai,
    formatcom=\color{white}
}
\usepackage{etoolbox}
\makeatletter
% replace \medskip before and after the box with nothing, i.e., remove it
\patchcmd{\minted@colorbg}{\medskip}{}{}{}
\patchcmd{\endminted@colorbg}{\medskip}{}{}{}
\makeatother

\renewcommand{\theFancyVerbLine}{\textcolor{black}{\arabic{FancyVerbLine}}}

\usepackage{pifont}
\newcommand{\cmark}{\ding{51}}%
\newcommand{\xmark}{\ding{55}}%

\newenvironment{changemargin}[2]{%
    \begin{list}{}{%
            \setlength{\topsep}{0pt}%
            \setlength{\leftmargin}{#1}%
            \setlength{\rightmargin}{#2}%
            \setlength{\listparindent}{\parindent}%
            \setlength{\itemindent}{\parindent}%
            \setlength{\parsep}{\parskip}%
        }%
        \item[]}{\end{list}}


\usepackage{csquotes}

% Title
\title{Visualisierung}
\author{Prof. Dr. Lukas Iffländer}
\institute{HTW Dresden}
\date{}
\usepackage{svg}

% Begin document
\begin{document}

% Title slide
\begin{frame}
    \titlepage
\end{frame}

\section{Noch etwas NumPy}

\begin{frame}{Berechnungen}
    \inputminted{python}{src/numpy_calc_1_dim.py}

\end{frame}

\begin{frame}{Berechnungen und Parametervariation}{Funktionen}
    \inputminted[firstline=5, lastline=17]{python}{src/numpy_calc_1_dim_func.py}

\end{frame}

\begin{frame}{Zwei zweidimensionale Numpy-Arrays zur Variation zweier Veränderlicher}
    \begin{exampleblock}{Elementweise Verknüpfung von Matritzen}
        \inputminted{python}{src/numpy_calc_2_dim.py}
        Als Ergebnis entsteht eine 3x3 Matrix der Energie in Nm für alle Kombinationen aus Masse (1000,1500 und 2000 kg) und Geschwindigkeiten (20, 40 und 60 m/s).

    \end{exampleblock}
\end{frame}

\begin{frame}{Parametervariation}
    \begin{columns}
        \begin{column}{0.6\textwidth}
            \inputminted[firstline=5,lastline=16]{python}{src/numpy_parameter_variation.py}
        \end{column}
        \begin{column}{0.4\textwidth}
            \begin{block}{Tipp}
                Die Koordinatenmatrizen müssen nicht per Hand getippt werden, sondern lassen sich aus Vektoren mit dem äußeren Vektorprodukt berechnen.
            \end{block}
        \end{column}
    \end{columns}

\end{frame}

\begin{frame}{Parametervariation}{Meshgrid}
    \begin{columns}
        \begin{column}{0.6\textwidth}
            \inputminted[firstline=3]{python}{src/numpy_meshgrid.py}
        \end{column}
        \begin{column}{0.4\textwidth}
            Das geht auch einfacher in einem Schritt:
            \begin{block}{\texttt{numpy.meshgrid}}
                \texttt{meshgrid:} erzeugt zwei Koordinatenmatrizen,
                \begin{enumerate}
                    \item Vektor wird horizontal in 1. Matrix variiert
                    \item Vektor wird vertikal in 2. Matrix variiert
                \end{enumerate}
            \end{block}
        \end{column}
    \end{columns}

\end{frame}

\section{Matplotlib}

\begin{frame}{Was ist Matplotlib?}
    \begin{itemize}
        \item Matplotlib ist eine Bibliothek zur Visualisierung von Daten in Python.
        \item Sie ermöglicht die Erstellung von 2D- und 3D-Diagrammen, Plots und Grafiken.
        \item Matplotlib ist besonders nützlich für die Analyse und Präsentation von Daten in wissenschaftlichen und technischen Anwendungen.
        \item Die darzustellenden Werte sind in der Regel in Numpy-Arrays gespeichert.
        \item Matplotlib ermöglicht den Export von Grafiken in verschiedenen Formaten wie PNG, PDF, SVG und anderen.
    \end{itemize}
\end{frame}

\begin{frame}{Minimalbeispiel}
    \begin{columns}
        \begin{column}{0.5\textwidth}
            \inputminted{python}{src/plt_sin.py}
        \end{column}
        \begin{column}{0.5\textwidth}
            \includegraphics[width=\textwidth]{fig/plt_sin.pdf}
        \end{column}
    \end{columns}

\end{frame}

\begin{frame}{Beschriftungen}
    \begin{columns}
        \begin{column}{0.5\textwidth}
            \inputminted[firstline=4]{python}{src/plt_labels.py}
        \end{column}
        \begin{column}{0.5\textwidth}
            \includegraphics[width=\textwidth]{fig/plt_labels.pdf}
        \end{column}
    \end{columns}

\end{frame}

\begin{frame}{Mehrere Kurven}
    \begin{columns}
        \begin{column}{0.5\textwidth}
            \inputminted[firstline=4]{python}{src/plt_multiple.py}
        \end{column}
        \begin{column}{0.5\textwidth}
            \includegraphics[width=\textwidth]{fig/plt_multiple.pdf}
        \end{column}
    \end{columns}
\end{frame}

\begin{frame}{Zeitreihe}{Kontinuierliche Darstellung}
    \begin{columns}
        \begin{column}{0.5\textwidth}
            \inputminted[firstline=4]{python}{src/plt_timeseries_cont.py}
        \end{column}
        \begin{column}{0.5\textwidth}
            \includegraphics[width=\textwidth]{fig/plt_timeseries_cont.pdf}
        \end{column}
    \end{columns}
\end{frame}

\begin{frame}{Zeitreihe}{Diskrete Darstellung}
    \begin{columns}
        \begin{column}{0.5\textwidth}
            \inputminted[firstline=4]{python}{src/plt_timeseries_disc_print.py}
        \end{column}
        \begin{column}{0.5\textwidth}
            \includegraphics[width=\textwidth]{fig/plt_timeseries_disc.pdf}
        \end{column}
    \end{columns}
\end{frame}

\begin{frame}{Subplots}
    \begin{columns}
        \begin{column}{0.6\textwidth}
            \small
            \inputminted[firstline=4]{python}{src/plt_subplot.py}
        \end{column}
        \begin{column}{0.4\textwidth}
            \includegraphics[width=\textwidth]{fig/plt_subplot.pdf}
        \end{column}
    \end{columns}
\end{frame}

\begin{frame}{3D-Plots}{Ebene}
    \begin{columns}
        \begin{column}{0.6\textwidth}
            \vspace{-\baselineskip}
            \small
            \inputminted[firstline=7]{python}{src/plt_3d_plane.py}
        \end{column}
        \begin{column}{0.4\textwidth}
            \includegraphics[width=\textwidth]{fig/plt_3d_plane.pdf}
        \end{column}
    \end{columns}
\end{frame}


\begin{frame}{3D-Plots}{Cowboy-Hut}
    \begin{columns}
        \begin{column}{0.6\textwidth}
            \vspace{-\baselineskip}
            \small
            \inputminted[firstline=7]{python}{src/plt_3d_cowboy.py}
        \end{column}
        \begin{column}{0.4\textwidth}
            \includegraphics[width=\textwidth]{fig/plt_3d_cowboy.pdf}
        \end{column}
    \end{columns}
\end{frame}

\begin{frame}{Ein- und Ausgabe}
    \begin{columns}
        \begin{column}{0.5\textwidth}
            \begin{itemize}
                \item nichtflüchtig gespeichertes Datenobjekt das von Programmen gelesen bzw. geschrieben werden kann
                \item wird gespeichert im Dateisystem, das oft Teil des Betriebssystems ist (Windows, Linux, IOS, Android),
                \item Vom Dateisystem wird die Zuordnung von Namen, die Platzierung in Verzeichnissen, die Zuordnung von Eigenschaften (Zugriffsdatum und -Zeit, Eigentümer, Berechtigungen u.a.) vorgenommen
            \end{itemize}
        \end{column}
        \begin{column}{0.5\textwidth}
        \end{column}
    \end{columns}
\end{frame}

\end{document}