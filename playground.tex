\documentclass[xelatex,aspectratio=169]{beamer}

\hfuzz=10pt
\vfuzz=10pt

% Theme
\usetheme{htw}
\setbeamertemplate{navigation symbols}{}
\setbeamertemplate{theorems}[numbered]
\setbeamercovered{transparent}

%\logo{\includegraphics[height=0.5cm]{HTWD_color.png}}

% Packages
\usepackage{polyglossia}
\setmainlanguage{german}
\setotherlanguage{english}
\usepackage{newunicodechar}
\newfontfamily\symbolfont{Noto Color Emoji}
\usepackage{diagbox}

\usepackage[bigfiles]{pdfbase}
\ExplSyntaxOn
\NewDocumentCommand\embedvideo{smm}{
\group_begin:
\leavevmode
\tl_if_exist:cTF{file_\file_mdfive_hash:n{#3}}{
    \tl_set_eq:Nc\video{file_\file_mdfive_hash:n{#3}}
}{
    \IfFileExists{#3}{}{\GenericError{}{File~`#3'~not~found}{}{}}
    \pbs_pdfobj:nnn{}{fstream}{{}{#3}}
    \pbs_pdfobj:nnn{}{dict}{
        /Type/Filespec/F~(#3)/UF~(#3)
        /EF~<</F~\pbs_pdflastobj:>>
    }
    \tl_set:Nx\video{\pbs_pdflastobj:}
    \tl_gset_eq:cN{file_\file_mdfive_hash:n{#3}}\video
}
%
\pbs_pdfobj:nnn{}{dict}{
    /Type/RichMediaInstance/Subtype/Video
    /Asset~\video
    /Params~<</FlashVars (
    source=#3&
    skin=SkinOverAllNoFullNoCaption.swf&
    skinAutoHide=true&
    skinBackgroundColor=0x5F5F5F&
    skinBackgroundAlpha=0.75
    )>>
}
%
\pbs_pdfobj:nnn{}{dict}{
/Type/RichMediaConfiguration/Subtype/Video
/Instances~[\pbs_pdflastobj:]
}
%
\pbs_pdfobj:nnn{}{dict}{
/Type/RichMediaContent
/Assets~<<
/Names~[(#3)~\video]
>>
/Configurations~[\pbs_pdflastobj:]
}
\tl_set:Nx\rmcontent{\pbs_pdflastobj:}
%
\pbs_pdfobj:nnn{}{dict}{
    /Activation~<<
    /Condition/\IfBooleanTF{#1}{PV}{XA}
    /Presentation~<</Style/Embedded>>
    >>
    /Deactivation~<</Condition/PI>>
}
%
\hbox_set:Nn\l_tmpa_box{#2}
\tl_set:Nx\l_box_wd_tl{\dim_use:N\box_wd:N\l_tmpa_box}
\tl_set:Nx\l_box_ht_tl{\dim_use:N\box_ht:N\l_tmpa_box}
\tl_set:Nx\l_box_dp_tl{\dim_use:N\box_dp:N\l_tmpa_box}
\pbs_pdfxform:nnnnn{1}{1}{}{}{\l_tmpa_box}
%
\pbs_pdfannot:nnnn{\l_box_wd_tl}{\l_box_ht_tl}{\l_box_dp_tl}{
    /Subtype/RichMedia
    /BS~<</W~0/S/S>>
    /Contents~(embedded~video~file:#3)
    /NM~(rma:#3)
    /AP~<</N~\pbs_pdflastxform:>>
    /RichMediaSettings~\pbs_pdflastobj:
    /RichMediaContent~\rmcontent
}
\phantom{#2}
\group_end:
}
\ExplSyntaxOff


\usepackage{graphicx}
\usepackage[export]{adjustbox}
\usepackage{animate}
%\usepackage[dvipdfmx]{movie15_dvipdfmx}
\usepackage{media9}
\usepackage{tabularx}
\usepackage{colortbl}
\usepackage{booktabs}
\usepackage{makecell}
\usepackage{ltablex}
\usepackage{array}
\usepackage{multirow}
\usepackage{amsmath}
\usepackage{amsthm}
\usepackage{mathtools}

\DeclarePairedDelimiter\ceil{\lceil}{\rceil}
\DeclarePairedDelimiter\floor{\lfloor}{\rfloor}
%\renewcommand{\arraystretch}{1.5}
\newcolumntype{L}[1]{>{\raggedright\let\newline\\\arraybackslash\hspace{0pt}}p{#1}}
\newcolumntype{C}[1]{>{\centering\let\newline\\\arraybackslash\hspace{0pt}}p{#1}}
\newcolumntype{R}[1]{>{\raggedleft\let\newline\\\arraybackslash\hspace{0pt}}p{#1}}
%\renewcommand\thesatz{\arabic{section}.\arabic{theorem}}
\makeatletter
\@addtoreset{theorem}{lecture}
\makeatother

\newtheorem{satz}{Satz}[section]
\newtheorem{lem}{Lemma}[section]
\newtheorem{beh}{Behauptung}[section]
\newtheorem{define}{Definition}[section]
\numberwithin{equation}{section}
\usepackage{ragged2e}
\usepackage{etoolbox}

\usepackage{color}
\usepackage{colortbl}
\definecolor{hellgrau}{rgb}{0.85,0.85,0.85}
\definecolor{hellrot}{rgb}{1,0.7,0.7}

\usepackage{tikz}
\usetikzlibrary{shapes,arrows.meta,calc,arrows,positioning,patterns,tikzmark,overlay-beamer-styles}
%\usepackage{tikz-uml}
\usepackage{pgfplots}  % for elliptic curves (part 8)
\pgfplotsset{compat=1.18}
\usepackage{pgffor}
\usepackage{pgfmath-xfp}
\tikzset{>=latex}
\tikzset{
    invisible/.style={opacity=0},
    visible on/.style={alt={#1{}{invisible}}},
    alt/.code args={<#1>#2#3}{%
            \alt<#1>{\pgfkeysalso{#2}}{\pgfkeysalso{#3}} % \pgfkeysalso doesn't change the path
        },
}

\usepackage{paralist}

\usepackage{url}
\def\UrlBreaks{\do\/\do-}
\PassOptionsToPackage{hyphens}{url}\usepackage{hyperref}

\usepackage[normalem]{ulem} % gestrichelte Unterstreichung (\dashuline{})
\usepackage{cancel}

\makeatletter
\renewcommand{\itemize}[1][]{%
    \beamer@ifempty{#1}{}{\def\beamer@defaultospec{#1}}%
    \ifnum \@itemdepth >2\relax\@toodeep\else
        \advance\@itemdepth\@ne
        \beamer@computepref\@itemdepth% sets \beameritemnestingprefix
        \usebeamerfont{itemize/enumerate \beameritemnestingprefix body}%
        \usebeamercolor[fg]{itemize/enumerate \beameritemnestingprefix body}%
        \usebeamertemplate{itemize/enumerate \beameritemnestingprefix body begin}%
        \list
        {\usebeamertemplate{itemize \beameritemnestingprefix item}}
        {\def\makelabel##1{%
                {%
                        \hss\llap{{%
                                    \usebeamerfont*{itemize \beameritemnestingprefix item}%
                                    \usebeamercolor[fg]{itemize \beameritemnestingprefix item}##1}}%
                    }%
            }%
        }
    \fi%
    \beamer@cramped%
    \justifying% NEW
    %\raggedright% ORIGINAL
    \beamer@firstlineitemizeunskip%
}
\makeatother

\apptocmd{\frame}{}{\justifying}{}

\renewcommand\theadfont{\bfseries\sffamily}
\usepackage{ragged2e}
\usepackage{newpxtext}

\setsansfont{opensans}[
    Scale=MatchLowercase,
    UprightFont=*-regular,
    BoldFont=*-bold,
    ItalicFont=*-italic,
    BoldItalicFont=*-bolditalic,
]

% Title
\usepackage[usetransparent=false]{svg}
% Import references
\usepackage[backend=biber,style=numeric,sorting=none]{biblatex}
\addbibresource{references.bib}

%\AtBeginSection[]{
%  \begin{frame}
%    \vfill
%    \centering
%    \begin{beamercolorbox}[sep=8pt,center,shadow=true,rounded=true]{title}
%      \usebeamerfont{title}\thesection.~\secname\par%
%    \end{beamercolorbox}
%    \vfill
%  \end{frame}
%}

\makeatletter
\newenvironment{noheadline}{
    \setbeamertemplate{headline}{}
    \addtobeamertemplate{frametitle}{\vspace*{-0.9\baselineskip}}{}
}{}
\makeatother


\usepackage{xcolor}
\usepackage{algorithm}
\usepackage[linesnumbered,ruled,lined,commentsnumbered,algo2e,ngerman,ngermankw]{algorithm2e}
\usepackage{algorithmic}
\usepackage{caption}
\usepackage[newfloat]{minted}
\captionsetup[listing]{position=top}
\definecolor{mintedbg}{HTML}{282828}
\setminted{
    breaklines=true,
    bgcolor=mintedbg,
    style=monokai,
    formatcom=\color{white}
}
\usepackage{etoolbox}
\makeatletter
% replace \medskip before and after the box with nothing, i.e., remove it
\patchcmd{\minted@colorbg}{\medskip}{}{}{}
\patchcmd{\endminted@colorbg}{\medskip}{}{}{}
\makeatother

\renewcommand{\theFancyVerbLine}{\textcolor{black}{\arabic{FancyVerbLine}}}

\usepackage{pifont}
\newcommand{\cmark}{\ding{51}}%
\newcommand{\xmark}{\ding{55}}%

\newenvironment{changemargin}[2]{%
    \begin{list}{}{%
            \setlength{\topsep}{0pt}%
            \setlength{\leftmargin}{#1}%
            \setlength{\rightmargin}{#2}%
            \setlength{\listparindent}{\parindent}%
            \setlength{\itemindent}{\parindent}%
            \setlength{\parsep}{\parskip}%
        }%
        \item[]}{\end{list}}


\usepackage{csquotes}

% Title
\title{Algorithmen}
\author{Prof. Dr. Lukas Iffländer}
\institute{HTW Dresden}
\date{}
\usepackage{svg}

% Begin document
\begin{document}

\begin{frame}{Strings in Python}{Grundlagen}
    \inputminted{python}{src/strings_hello.py}
    \begin{itemize}
        \item Bei einer solchen Zuweisung wird der Datentyp \texttt{str} (kurz für String) verwendet.
        \item Python verwendet derzeit UTF-16 intern (eine Migration auf UTF-8 in \enquote{einer zukünftigen Version} ist geplant).
        \item Einzelnes Zeichen: \mintinline{python}|z = 'x'|
        \item Zeichenkette: \mintinline{python}|s = 'Achtung!'|
        \item Mehrzeilige Strings
              \inputminted{python}{src/strings_multiline.py}
    \end{itemize}
\end{frame}

\begin{frame}{Strings in Python}{Umrechnen in ASCII}
    Funktionen für einzelne Zeichen (Strings der Länge 1):
    \inputminted{python}{src/strings_conversion.py}

\end{frame}

\begin{frame}{Strings in Python}{Ausgewählte Funktionen und Methoden}
    \inputminted{python}{src/strings_functions.py}
\end{frame}

\begin{frame}{Strings in Python}{Vergleichen}
    \inputminted{python}{src/strings_comparison.py}
    \begin{itemize}
        \item Strings können mit den Operatoren \texttt{==}, \texttt{!=}, \texttt{<}, \texttt{<=}, \texttt{>} und \texttt{>=} verglichen werden.
        \item Dabei wird lexikographisch verglichen (wie im Wörterbuch).
        \item Die Vergleiche sind case-sensitive.
    \end{itemize}
\end{frame}

\begin{frame}{Strings in Python}{Teilzeichenketten (Substrings) zählen}
    \inputminted{python}{src/strings_substring_count.py}
    \begin{itemize}
        \item Die Methode \texttt{count()} zählt, wie oft ein Teilstring in einem String vorkommt.
        \item Die Methode ist case-sensitive.
    \end{itemize}
\end{frame}

\begin{frame}{Strings in Python}{Teilzeichenketten (Substrings) finden}
    \inputminted{python}{src/strings_substring_find.py}
    \begin{itemize}
        \item Die Methode \texttt{find()} gibt den Index des ersten Vorkommens eines Teilstrings zurück.
        \item Wenn der Teilstring nicht gefunden wird, gibt die Methode \texttt{-1} zurück.
        \item Die Methode ist case-sensitive.
        \item Die Methode \texttt{rfind()} gibt den Index des letzten Vorkommens eines Teilstrings zurück.
    \end{itemize}
\end{frame}

\begin{frame}{Strings in Python}{Indizierung}
    \inputminted[lastline=3]{python}{src/strings_index.py}
    \begin{itemize}
        \item Strings sind indizierbar und können über Indizes angesprochen werden.
        \item Der Index ist 0-basiert, d.h. der erste Buchstabe hat den Index 0.
        \item Negative Indizes zählen von hinten, d.h. der letzte Buchstabe hat den Index -1.
    \end{itemize}
    \inputminted[firstline=5]{python}{src/strings_index.py}
    \begin{itemize}
        \item Durch die Indizierung mit \texttt{[start:end]} kann ein Teilstring erstellt werden.
        \item Der \texttt{start}-Index ist inklusiv, der \texttt{end}-Index exklusiv (wie bei \texttt{range()}).
        \item Der \texttt{end}-Index kann weggelassen werden, dann wird bis zum Ende des Strings geschnitten.
        \item Der \texttt{start}-Index kann weggelassen werden, dann wird vom Anfang des Strings geschnitten.
    \end{itemize}

\end{frame}

\begin{frame}{Strings in Python}{Indizierung -- Beispiele}
    \inputminted{python}{src/strings_index_examples.py}
\end{frame}

\begin{frame}{Strings in Python}{Zusammensetzung}
    \inputminted{python}{src/strings_concat.py}
    \begin{itemize}
        \item Strings können durch den Operator \texttt{+} zusammengefügt werden.
    \end{itemize}
\end{frame}

\begin{frame}{Strings in Python}{Zusammensetzung löngerer Strings -- Alte Methode}
    \begin{columns}
        \begin{column}{0.5\textwidth}
            \begin{itemize}
                \item Programme erfordern das Zusammensetzen von Ausgaben aus numerischen Werten und Strings.
                \item Numerische Werte werden mit \texttt{str()} in Strings umgewandelt und mit \texttt{+} zusammengefügt.
            \end{itemize}
        \end{column}
        \begin{column}{0.5\textwidth}
            \inputminted{python}{src/strings_concat_long_old.py}
        \end{column}
    \end{columns}

\end{frame}

\begin{frame}{Strings in Python}{Zusammensetzung löngerer Strings -- Neue Methode}
    \inputminted{python}{src/strings_concat_long_new.py}
    \begin{itemize}
        \item Die neue Methode verwendet f-Strings (ab Python 3.6).
        \item Variablen werden in geschweifte Klammern gesetzt.
        \item Der String wird mit einem \texttt{f} vorangestellt.
    \end{itemize}
\end{frame}

\begin{frame}{Strings in Python}{Ausgabeformatierung mit f-Strings}
    \inputminted{python}{src/strings_concat_long_new_float.py}
    \begin{itemize}
        \item Mit f-Strings können auch Formatierungen vorgenommen werden.
        \item Die Formatierung erfolgt in geschweiften Klammern.
    \end{itemize}
\end{frame}



% End document
\end{document}